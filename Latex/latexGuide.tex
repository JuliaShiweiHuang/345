\documentclass[a4paper,11pt]{article}
\usepackage{textcomp}
\begin{document}
% Example 1
\ldots when Einstein introduced his formula \begin{tabular}{@{} l @{}}
\hline
no leading space\\
\hline
\end{tabular}
\begin{tabular}{|r|l|}
\hline
7C0 & hexadecimal \\
3700 & octal \\ \cline{2-2}
11111000000 & binary \\
\hline \hline
1984 & decimal \\
\hline
\end{tabular}
A typographical rule of thumb
for the line length is:
\begin{tabular}{c r @{.} l}
Pi expression &
\multicolumn{2}{c}{Value} \\
\hline
$\pi$ & 3&1416 \\
$\pi^{\pi}$ & 36&46 \\
$(\pi^{\pi})^{\pi}$ & 80662&7 \\
\end{tabular}
Add $a$ squared and $b$ squared
to get $c$ squared. Or, using
a more mathematical approach:
$a^2 + b^2 = c^2$
Figure~\ref{white} is an example of Pop-Art.
\TeX{} is pronounced as
$\tau\epsilon\chi$\\[5pt]
100~m$^{3}$ of water\\[5pt]
100$ m^3$ of water 
This comes from my $\heartsuit$
\begin{figure}[!hbtp]
\makebox[\textwidth]{\framebox[5cm]{\rule{0pt}{5cm}}}
\caption{Five by Five in Centimetres.\label{white}}
\end{figure}


\begin{quote}
On average, no line should
be longer than 66 characters.
\end{quote}
This is why \LaTeX{} pages have
such large borders by default
and also why multicolumn print
is used in newspapers.
The \verb|\ldots| command \ldots
\begin{verbatim}
10 PRINT "HELLO WORLD ";
20 GOTO 10
#include <iostream>
#include <iomanip>
#include <fstream>

using namespace std;

class monkey {
public:
    bool hungry; //1(a) a bool hungry.
    monkey(){cout << "Hello, I'm the constructor from class monkey that takes no input." << endl;}; //1(b) a constructor that takes no input. 
    monkey (bool something){cout << "Hello, I'm the constructor from class monkey that takes a bool as input." << endl;}; //1(c) a constructor that takes a bool as its input.
    ~monkey(){cout << "Hello, I'm the descructor from class monkey." << endl;}; //1(d) a destructor. \hfill \break
    void eatBanana () {cout << "Hello, I am the function eatBanana that takes no input." << endl; }; //1(e) a function eatBanana that takes no input.
    virtual void monkeyAround () {cout << "Hello, I'm the virtual function from class monkey that takes no input." << endl;}; // a virtual function monkeyAround that takes no input.
};
\end{verbatim}

\begin{tabular}{|r|l|}
\hline
7C0 & hexadecimal \\
3700 & octal \\ \cline{2-2}
11111000000 & binary \\
\hline \hline
1984 & decimal \\
\hline
\end{tabular}


\begin{equation}
e = m \cdot c^2 \; ,
\end{equation}
which is at the same time the most widely known
A reference to this subsection
\label{sec:this} looks like:
‘‘see section~\ref{sec:this} on
page~\pageref{sec:this}.’’ \underline{something}
and the least well understood physical formula. $0$, $1$, and $-1$
% Example 2
\ldots from which follows\\* Kirchhoff’s current law:
\begin{equation}
\sum_{k=1}^{n} I_k = 0 \; .
\end{equation}
Kirchhoff’s voltage law can be derived \ldots
% Example 3
\ldots which has several advantages.
\flushleft
\begin{enumerate}
\item You can nest the list
environments to your taste:
\begin{itemize}
\item But it might start to
look silly.
\item[-] With a dash.
\end{itemize}
\item Therefore remember:
\begin{description}
\item[Stupid] things will not
become smart because they are
in a list.
\item[Smart] things, though,
can be presented beautifully
in a list.
\end{description}
\end{enumerate}
\begin{equation}
I_D = I_F - I_R
\end{equation}
is the core of a very different transistor model. read/writer \texteuro 3 ead\slash write
\hyphenation{FORTRAN Hy-phen-a-tion}  \ldots 
\end{document}